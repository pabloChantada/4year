\section{Captura y análisis de tráfico}


\begin{itemize}
    \item ¿Qué podrías saber sobre tu ordenador (navegador, sistema operativo, IP, ...)
    \item ¿Qué información podrías obtener sobre tu actividad en Internet (páginas visitadas, servicios usados, ...)?
    \item ¿Sería adecuado utilizar esta captura, tal como está, para crear un dataset de acceso público? Razona la respuesta
\end{itemize}

De nuestro ordenador obtenemos esta informacion por ejemplo:

\begin{figure}[H]
    \centering
    \includegraphics[width=0.8\textwidth]{imgs/asus.png}
    \caption{Captura de tráfico con Wireshark}
    \label{fig:wireshark}
\end{figure}

Indicando la IP de nuestro ordenador y que utilizamos ASUS, que corresponde con la tarejeta grafica.

En cuanto a la actividad en Internet, se pueden observar las páginas visitadas, los servicios usados, etc. 
Por ejemplo, observamos que accedimos a google y steam: 

\begin{figure}[H]
    \centering
    \includegraphics[width=0.8\textwidth]{imgs/steam.png}
    \caption{Captura de tráfico con Wireshark}
    \label{fig:wireshark2}
\end{figure}